%\title{SlugCam submission for PerCom 2015}
%% percomPaper.tex
%% Draft version: 1.0
%% 8/25/2014
%% Authors:
%% Kevin Abas
%% Leland Miller
%% Katia Obraczka
%% see SlugCam's Organization page: https://github.com/SlugCam



%%*************************************************************************
%% Legal Notice:
%% This code is offered as-is without any warranty either expressed or
%% implied; without even the implied warranty of MERCHANTABILITY or
%% FITNESS FOR A PARTICULAR PURPOSE! 
%% User assumes all risk.
%% In no event shall IEEE or any contributor to this code be liable for
%% any damages or losses, including, but not limited to, incidental,
%% consequential, or any other damages, resulting from the use or misuse
%% of any information contained here.
%%
%% All comments are the opinions of their respective authors and are not
%% necessarily endorsed by the IEEE.
%%
%% This work is distributed under the LaTeX Project Public License (LPPL)
%% ( http://www.latex-project.org/ ) version 1.3, and may be freely used,
%% distributed and modified. A copy of the LPPL, version 1.3, is included
%% in the base LaTeX documentation of all distributions of LaTeX released
%% 2003/12/01 or later.
%% Retain all contribution notices and credits.
%% ** Modified files should be clearly indicated as such, including  **
%% ** renaming them and changing author support contact information. **
%%
%% File list of work: IEEEtran.cls, IEEEtran_HOWTO.pdf, bare_adv.tex,
%%                    bare_conf.tex, bare_jrnl.tex, bare_jrnl_compsoc.tex,
%%                    bare_jrnl_transmag.tex
%%*************************************************************************


%
\documentclass[journal,transmag]{IEEEtran}
%



% Some very useful LaTeX packages include:
% (uncomment the ones you want to load)


% *** MISC UTILITY PACKAGES ***
%
%\usepackage{ifpdf}
%
% \ifpdf
%   % pdf code
% \else
%   % dvi code
% \fi


% *** CITATION PACKAGES ***
%
%\usepackage{cite}


% *** GRAPHICS RELATED PACKAGES ***
%
\ifCLASSINFOpdf
  % \usepackage[pdftex]{graphicx}
  % declare the path(s) where your graphic files are
  % \graphicspath{{../pdf/}{../jpeg/}}
  % and their extensions so you won't have to specify these with
  % every instance of \includegraphics
  % \DeclareGraphicsExtensions{.pdf,.jpeg,.png}
\else
  % or other class option (dvipsone, dvipdf, if not using dvips). graphicx
  % will default to the driver specified in the system graphics.cfg if no
  % driver is specified.
  % \usepackage[dvips]{graphicx}
  % declare the path(s) where your graphic files are
  % \graphicspath{{../eps/}}
  % and their extensions so you won't have to specify these with
  % every instance of \includegraphics
  % \DeclareGraphicsExtensions{.eps}
\fi


% *** MATH PACKAGES ***
%
%\usepackage[cmex10]{amsmath}


% *** SPECIALIZED LIST PACKAGES ***
%
%\usepackage{algorithmic}


% *** ALIGNMENT PACKAGES ***
%
%\usepackage{array}


% *** SUBFIGURE PACKAGES ***
%\ifCLASSOPTIONcompsoc
%  \usepackage[caption=false,font=normalsize,labelfont=sf,textfont=sf]{subfig}
%\else
%  \usepackage[caption=false,font=footnotesize]{subfig}
%\fi


% *** FLOAT PACKAGES ***
%
%\usepackage{fixltx2e}
%
%\usepackage{stfloats}
%
%\usepackage{dblfloatfix}
%
%\ifCLASSOPTIONcaptionsoff
%  \usepackage[nomarkers]{endfloat}
% \let\MYoriglatexcaption\caption
% \renewcommand{\caption}[2][\relax]{\MYoriglatexcaption[#2]{#2}}
%\fi
%
% For subfig.sty:
% \let\MYorigsubfloat\subfloat
% \renewcommand{\subfloat}[2][\relax]{\MYorigsubfloat[]{#2}}
%


% *** PDF, URL AND HYPERLINK PACKAGES ***
%
%\usepackage{url}


\begin{document}


%% ***********************  THE TITLE **********************************



\title{Solar Smart Camera Networks for Ubiquitous Outdoor Video Monitoring}



%% ***********************  AUTHORS ************************************



\author{\IEEEauthorblockN{Kevin Abas\IEEEauthorrefmark{1},
Leland Miller \IEEEauthorrefmark{1},
Katia Obraczka\IEEEauthorrefmark{1}}
\IEEEauthorblockA{\IEEEauthorrefmark{1}School of Engineering,
University of California Santa Cruz, CA 95064 USA}
}



%% ***********************  Abstract ************************************



\IEEEtitleabstractindextext{%
\begin{abstract}

Abstract: Blah Blah Solar camera, our contributions are, etc. \cite{DSPcam}

\end{abstract}



%% ********************** IEEE KEYWORDS ********************************



\begin{IEEEkeywords}
Real-time systems and embedded
systems,Interactive systems, Web services, Computer vision, Human safety
\end{IEEEkeywords}}

% make the title area
\maketitle

\IEEEdisplaynontitleabstractindextext
% \IEEEdisplaynontitleabstractindextext has no effect when using
% compsoc or transmag under a non-conference mode.

% For peerreview papers, this IEEEtran command inserts a page break and
% creates the second title. It will be ignored for other modes.
\IEEEpeerreviewmaketitle



%%
%% *********************  BEGIN PAPER **********************************
%%



\section{Introduction}


 \IEEEPARstart{I}{}ntroduction here discussing the paper
 Bring up both areas we'll be discussing



%%
%% *********************  Section 2 **********************************
%%



\section{Motivation}

Explain campus safety application possibly? Potential use cases




%%
%% *********************  Section 3 **********************************
%%



\section{Overview of SlugCam}

	\subsection{Design choices}

		\subsubsection{Power awareness}

		\subsubsection{Application requirements}

		\subsubsection{Functionality}

	\subsection{System overview}



%%
%% *********************  Section 4 **********************************
%%



\section{Hardware system}

	\subsection{Organization}
    
	\subsection{Subsystems}
    
    \subsection{Explain why components were chosen}
    
    \subsection{solar/battery}



%%
%% *********************  Section 5 **********************************
%%



\section{Software system}

	\subsection{Node Operating system}
    
    	\subsubsection{Embedded design}
        
        \subsubsection{Power awareness}
        
	\subsection{Web Server Stack}
    	
       \subsubsection{MEAN stacks}
    
    \subsection{User interface}
    
        \subsubsection{Description of video file management system}
        
        

%%
%% *********************  Section 6 **********************************
%%



\section{System operation}
	
    \subsection{Duty Cycles}



%%
%% *********************  Section 7 **********************************
%%
 
 
 
\section{Deployment} 
	\subsection{ Outdoor power characterization and testing}
    	\subsubsection{experiments in lab with current sensor/multimeter analyzing the systems current consumption}
		
        \subsubsection{application use cases, both solar/battery performance tests and computer vision tests.}
        
		\subsubsection{solar/battery after}
        
    \subsection{ UI application testing}
    
    
    
%%
%% *********************  Section 8 **********************************
%%    



\section{Background and related work}



%%
%% *********************  Section 9 **********************************
%% 



\section{future work and conclusion}



%% *************************  END PAPER **********************************



% use section* for acknowledgements
\section*{Acknowledgment}


The authors would like to thank...


% Can use something like this to put references on a page
% by themselves when using endfloat and the captionsoff option.
\ifCLASSOPTIONcaptionsoff
  \newpage
\fi



%% ******************  REFERENCE SECTION **********************************



\bibliography{Bib.bib}{}
\bibliographystyle{unsrt}



%% ******************  AUTHOR BIOGRAPHIES *********************************




\begin{IEEEbiographynophoto}{Kevin Abas}
Biography text here.
\end{IEEEbiographynophoto}

\begin{IEEEbiographynophoto}{Leland Miller}
Biography text here.
\end{IEEEbiographynophoto}

\begin{IEEEbiographynophoto}{Katia Obraczka}
Biography text here.
\end{IEEEbiographynophoto}

% You can push biographies down or up by placing
% a \vfill before or after them. The appropriate
% use of \vfill depends on what kind of text is
% on the last page and whether or not the columns
% are being equalized.

%\vfill

% Can be used to pull up biographies so that the bottom of the last one
% is flush with the other column.
%\enlargethispage{-5in}




\end{document}

